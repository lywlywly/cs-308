\documentclass[10pt]{article}

\usepackage[margin=1in]{geometry}
% https://tex.stackexchange.com/questions/27802/set-noindent-for-entire-file
\setlength\parindent{0pt}
% https://tex.stackexchange.com/questions/9550/why-does-underlined-text-not-get-wrapped-once-it-hits-the-end-of-a-line
\usepackage{soul}
\usepackage{indentfirst}
\usepackage{amsmath,amsthm,amssymb,scrextend}
\DeclareMathOperator*{\argmax}{arg\,max}
\DeclareMathOperator*{\argmin}{arg\,min}
% \usepackage{fancyhdr}
% \setlength{\headheight}{14.49998pt}
% \pagestyle{fancy}
\usepackage{graphicx}

\newcommand{\cont}{\subseteq}
\usepackage{tikz}
\usepackage{pgfplots}
\pgfplotsset{width=10cm,compat=1.9}

\usepackage{amsmath}
\usepackage[mathscr]{euscript}
\let\euscr\mathscr \let\mathscr\relax
\usepackage[scr]{rsfso}
\usepackage{amsthm}
\usepackage{amssymb}
\usepackage{multicol}

\usepackage{algorithm}
\usepackage{algpseudocode}

\usepackage{listings}
\usepackage{xcolor}
\usepackage{float}
\definecolor{codegreen}{rgb}{0,0.6,0}
\definecolor{codegray}{rgb}{0.5,0.5,0.5}
\definecolor{codepurple}{rgb}{0.58,0,0.82}
\definecolor{backcolour}{rgb}{0.95,0.95,0.92}

\lstdefinestyle{mystyle}{
    backgroundcolor=\color{backcolour},   
    commentstyle=\color{codegreen},
    keywordstyle=\color{magenta},
    numberstyle=\tiny\color{codegray},
    stringstyle=\color{codepurple},
    basicstyle=\ttfamily\footnotesize,
    breakatwhitespace=false,         
    breaklines=true,                 
    captionpos=b,                    
    keepspaces=true,                 
    numbers=left,                    
    numbersep=5pt,                  
    showspaces=false,                
    showstringspaces=false,
    showtabs=false,                  
    tabsize=2
}

\lstset{style=mystyle}

\DeclareMathOperator{\arcsec}{arcsec}
\DeclareMathOperator{\arccot}{arccot}
\DeclareMathOperator{\arccsc}{arccsc}
\newcommand{\ddx}{\frac{d}{dx}}
\newcommand{\dfdx}{\frac{df}{dx}}
\newcommand{\ddxp}[1]{\frac{d}{dx}\left( #1 \right)}
\newcommand{\dydx}{\frac{dy}{dx}}
\let\ds\displaystyle
\newcommand{\intx}[1]{\int #1 \, dx}
\newcommand{\intt}[1]{\int #1 \, dt}
\newcommand{\defint}[3]{\int_{#1}^{#2} #3 \, dx}
\newcommand{\imp}{\Rightarrow}
\newcommand{\un}{\cup}
\newcommand{\inter}{\cap}
\newcommand{\ps}{\mathscr{P}}
\newcommand{\set}[1]{\left\{ #1 \right\}}
\newtheorem*{sol}{Solution}
\newtheorem*{claim}{Claim}
\newtheorem{problem}{Problem}
\def\mathbi#1{\textbf{\em #1}}
\begin{document}

% \lhead{Stats 303}
% \chead{Luyao Wang lw337}
% \rhead{\today}

\begin{center}
  {\Large \bf COMPSCI 308: Design and Analysis of Algorithms Homework 3}
  \vspace{2mm}

  {\bf Luyao Wang}

  {\today}
\end{center}

\section*{1. Greedy Algorithms}
\subsection*{(a) Pseudocode}

\begin{algorithm}[H]
  \caption{Greedy Range Cover Algorithm}
  \begin{algorithmic}[1]
    \Function{GreedyRangeCover}{$A$, $\text{length}$}
    \State $rangeList \gets$ empty list
    \State $currentRangeStart \gets A[0]$
    \State $currentRangeEnd \gets A[0] + \text{length}$

    \For{$i \gets 1$ to $\text{length}(A)$}
    \If{$A[i] \leq currentRangeEnd$}
    \Comment{Add the last range to the list}
    \State \textbf{continue}
    \Else
    \Comment{Add the current range to the list and start a new range}
    \State $rangeList$.append([$currentRangeStart$, $currentRangeEnd$])
    \State $currentRangeStart \gets A[i]$
    \State $currentRangeEnd \gets A[i] + \text{length}$
    \EndIf
    \EndFor

    \State $rangeList$.append([$currentRangeStart$, $currentRangeEnd$])
    \State \textbf{return} $rangeList.length$
    \EndFunction
  \end{algorithmic}
\end{algorithm}

\subsection*{(b) Pseudocode + Optimality}

\begin{algorithm}[H]
  \caption{Two Room Activity Selection}
  \begin{algorithmic}[1]
    \Procedure{TwoRoomActivitySelection}{activities: list[Activity]}
    \State activities.sort(key=x => x.end) \Comment{Sort activities by ending time}
    \State A1 = [Activity(-inf, -inf)] \Comment{Activities assigned to room 1}
    \State A2 = [Activity(-inf, -)] \Comment{Activities assigned to room 2}

    \For{activity in activities}
    \If{activity.start $<$ A1[-1].end \textbf{ and } activity.start $<$ A2[-1].end}
    \State \textbf{continue} \Comment{Skip the activity}
    \ElsIf{activity.start $\geq$ A1[-1].end \textbf{ and } activity.start $<$ A2[-1].end}
    \State A1.append(activity) \Comment{Assign activity to room 1}
    \ElsIf{activity.start $<$ A1[-1].end \textbf{ and } activity.start $\geq$ A2[-1].end}
    \State A2.append(activity) \Comment{Assign activity to room 2}
    \ElsIf{activity.start $\geq$ A1[-1].end \textbf{ and } activity.start $\geq$ A2[-1].end}
    \If{A1[-1].end $\geq$ A2[-1].end}
    \Comment{Assign activity to the room with latest end time}
    \State A1.append(activity)
    \Else
    \State A2.append(activity)
    \EndIf
    \EndIf
    \EndFor

    \State \textbf{return} A1[1:], A2[1:] \Comment{Return activities after the dummy activity}
    \EndProcedure
  \end{algorithmic}
\end{algorithm}

The greedy algorithm employed in the Two Room Activity Selection problem selects the room for an activity based on the room with the \textbf{latest} end time. This strategy aims to replace the activity in the room with the latest ending time, as it maximizes the available "space" in the \textbf{other} room. By doing so, future intervals with early start times can fit into the schedule.

\section*{2. Graph Algorithms}

\subsection*{(a) Programming}


\begin{verbatim}
class Node:
    def __init__(self, data: int):
        self.data = data
        self.next: Node | None = None


class LinkedList:
    def __init__(self):
        self.head = None

    def append(self, data: int):
        new_node = Node(data)
        if not self.head:
            self.head = new_node
        else:
            curr_node = self.head
            while curr_node.next:
                curr_node = curr_node.next
            curr_node.next = new_node

    def __iter__(self):
        curr_node = self.head
        while curr_node:
            yield curr_node.data
            curr_node = curr_node.next


class GraphAdjacencyList:
    def __init__(self, v: int):
        self.v: int = v
        self.e: int = 0
        self._adj: list[LinkedList] = [LinkedList() for _ in range(v)]

    def add_edge(self, v: int, w: int):
        if 0 <= v < self.v and 0 <= w < self.v:
            self._adj[v].append(w)
            self.e += 1

    def V(self):
        return self.v

    def E(self):
        return self.e

    def adj(self, v: int) -> list[int]:
        return list(self._adj[v])

    @classmethod
    def from_adj_matrix(cls, graph: "GraphAdjacencyMatrix"):
        v = graph.V()
        adj_list = cls(v)

        for v in range(v):
            for w in graph.adj(v):
                adj_list.add_edge(v, w)

        return adj_list
        
        
class GraphAdjacencyMatrix:
    def __init__(self, v: int):
        self.v: int = v
        self.e: int = 0
        self._adj_matrix: list[list[int]] = [[0] * v for _ in range(v)]

    def add_edge(self, v: int, w: int):
        if 0 <= v < self.v and 0 <= w < self.v:
            self._adj_matrix[v][w] = 1
            self.e += 1

    def V(self):
        return self.v

    def E(self):
        return self.e

    def adj(self, v: int) -> list[int]:
        return [i for i, val in enumerate(self._adj_matrix[v]) if val == 1]

    @classmethod
    def from_adj_list(cls, graph: GraphAdjacencyList):
        v = graph.V()
        adj_list = cls(v)

        for v in range(v):
            for w in graph.adj(v):
                adj_list.add_edge(v, w)

        return adj_list
\end{verbatim}

I have implemented two common graph representations: adjacency matrix and adjacency list for a directed graph. Additionally, I provided a simple version of a linked list for the adjacency list representation.

The conversion methods for converting between the adjacency matrix and adjacency list representations are implemented as \textit{classmethods}.

The time complexity of the conversion algorithms is $O(V^2 + E)$.

\begin{itemize}
  \item {adj list to adj matrix:
        $O(V^2)$ for creating a matrix of shape (V, V) and initializing it with zeros.
        $O(E)$ because there are E edges that need to be created. The cost of creating an edge in the matrix is O(1).
        }
  \item {adj matrix to adj list:
        $O(V^2)$ for iterating over the adjacency matrix of shape (V, V).
        $O(E)$ because there are E edges that need to be created. The cost of creating an edge in the matrix is O(1).
        }
\end{itemize}

Therefore, the total time complexity for both conversion functions, from adjacency list to adjacency matrix and from adjacency matrix to adjacency list, is $O(V^2 + E)$.

\bigbreak

Below is the testing code and the running result.

\begin{verbatim}
def main():
    # testing adjacency list
    print("testing adjacency list")
    graph0 = GraphAdjacencyList(v=5)

    graph0.add_edge(0, 1)
    graph0.add_edge(0, 4)
    graph0.add_edge(1, 2)
    graph0.add_edge(1, 3)
    graph0.add_edge(2, 3)
    graph0.add_edge(3, 4)
    graph0.add_edge(4, 3)
    graph0.add_edge(4, 1)

    print(f"Number of vertices: {graph0.V()}")
    print(f"Number of edges: {graph0.E()}")
    
    for v in range(graph0.V()):
        print(f"Adjacent vertices of {v}: {graph0.adj(v)}")

    # testing adjacency matrix
    print("testing adjacency matrix")

    graph1 = GraphAdjacencyMatrix(v=5)
    graph1.add_edge(0, 1)
    graph1.add_edge(0, 4)
    graph1.add_edge(1, 2)
    graph1.add_edge(1, 3)
    graph1.add_edge(2, 3)
    graph1.add_edge(3, 4)
    graph1.add_edge(4, 3)
    graph1.add_edge(4, 1)

    print(f"Number of vertices: {graph1.V()}")
    print(f"Number of edges: {graph1.E()}")

    for v in range(graph1.V()):
        print(f"Adjacent vertices of {v}: {graph1.adj(v)}")

    # testing adj list 2 adj matrix
    print("testing adj list 2 adj matrix")
    graph2 = GraphAdjacencyMatrix.from_adj_list(graph0)

    print(f"Number of vertices: {graph2.V()}")
    print(f"Number of edges: {graph2.E()}")

    for v in range(graph2.V()):
        print(f"Adjacent vertices of {v}: {graph2.adj(v)}")

    # testing adj matrix 2 adj list
    print("testing adj matrix 2 adj list")
    graph3 = GraphAdjacencyList.from_adj_matrix(graph1)

    print(f"Number of vertices: {graph3.V()}")
    print(f"Number of edges: {graph3.E()}")

    for v in range(graph3.V()):
        print(f"Adjacent vertices of {v}: {graph3.adj(v)}")


main() if __name__ == "__main__" else None
\end{verbatim}

\begin{verbatim}
$ python3 graph.py
>>> testing adjacency list
>>> Number of vertices: 5
>>> Number of edges: 8
>>> Adjacent vertices of 0: [1, 4]
>>> Adjacent vertices of 1: [2, 3]
>>> Adjacent vertices of 2: [3]
>>> Adjacent vertices of 3: [4]
>>> Adjacent vertices of 4: [3, 1]
>>> testing adjacency matrix
>>> Number of vertices: 5
>>> Number of edges: 8
>>> Adjacent vertices of 0: [1, 4]
>>> Adjacent vertices of 1: [2, 3]
>>> Adjacent vertices of 2: [3]
>>> Adjacent vertices of 3: [4]
>>> Adjacent vertices of 4: [1, 3]
>>> testing adj list 2 adj matrix
>>> Number of vertices: 5
>>> Number of edges: 8
>>> Adjacent vertices of 0: [1, 4]
>>> Adjacent vertices of 1: [2, 3]
>>> Adjacent vertices of 2: [3]
>>> Adjacent vertices of 3: [4]
>>> Adjacent vertices of 4: [1, 3]
>>> testing adj matrix 2 adj list
>>> Number of vertices: 5
>>> Number of edges: 8
>>> Adjacent vertices of 0: [1, 4]
>>> Adjacent vertices of 1: [2, 3]
>>> Adjacent vertices of 2: [3]
>>> Adjacent vertices of 3: [4]
>>> Adjacent vertices of 4: [1, 3]
\end{verbatim}

\subsection*{(b) Pseudocode}

\begin{algorithm}
  \caption{Binary Label}
  \begin{algorithmic}[1]
    \Function{isBipartite}{G}
    \State \textbf{initialize} marked as empty boolean array
    \State \textbf{initialize} labels as empty boolean array
    \State \textbf{initialize} canBipartite as \textbf{true}
    \For{$s$ \textbf{in} vertices of $G$}
    \If{not marked[$s$]}
    \State \Call{dfs}{G, s, marked, labels}
    \EndIf
    \EndFor
    \State \textbf{return} canBipartite
    \EndFunction

    \Function{dfs}{G, v, marked, labels}
    \State marked[$v$] $\gets$ \textbf{true}
    \For{$w$ \textbf{in} neighbors of $v$ in $G$}
    \If{not marked[$w$]}
    \State labels[$w$] $\gets$ \textbf{not} labels[$v$]
    \State \Call{dfs}{G, w, marked, color}
    \ElsIf{labels[$w$] == labels[$v$]}
    \State canBipartite $\gets$ \textbf{false}
    \EndIf
    \EndFor
    \EndFunction
  \end{algorithmic}
\end{algorithm}

\end{document}